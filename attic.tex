\subsection{Algorithm extensions and other Applications}

Our current implementation was designed for the special case of
translating mappings into simple OWL axioms of either the SubClass or
Equivalence type. For other domains, richer OWL constructs would be
useful (for example, existential restrictions, to represent
relationships to members of classes, e.g. partOf). The general
approach can be applied here, but the main challenge will be the
expansion of the search space, requiring more sophisticated approaches
to prune and explore, e.g. MCMC.





From this we can construct a symmetric edge representation $E$ of $A^P$:

% Iverson notation, square brackets denote a function that yields true if size of set > 0
\begin{equation}
E_{ij} = [ C_i \sqsubseteq C_j] [ C_j \sqsubseteq C_i ]
\end{equation}

This yields 4 possibilities for any edge pair:

\begin{itemize}
\item[] $E_{ij} = 01$ if $C_i \sqsubset C_j$ aka \emph{Proper SubClassOf}, i.e. $C_i \sqsubseteq C_j, C_j \not\sqsubseteq C_i$
\item[] $E_{ij} = 10$ if $C_i \sqsupset C_j$ aka \emph{Proper SuperClassOf}, i.e.  $C_j \sqsubseteq C_i, C_i \not\sqsubseteq C_j$
\item[] $E_{ij} = 11$ if $C_i \equiv C_j$ aka \emph{EquivalentTo} i.e. $C_i \sqsubseteq C_j, C_j \sqsubseteq C_i$
\item[] $E_{ij} = 00$ if $C_i \parallel C_j$ i.e. $C_i \not\sqsubseteq C_j, C_i \not\sqsubseteq C_j$
\end{itemize}

We can rewrite \eqnref{JointLikelihood} replacing $A^P$ with $E$.
